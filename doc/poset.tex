%\documentclass[12pt,a4paper]{amsart}
\documentclass[10pt,a4paper]{article}
%\ifx\pdfpageheight\undefined\PassOptionsToPackage{dvips}{graphicx}\else%
%\PassOptionsToPackage{pdftex}{graphicx}
%\PassOptionsToPackage{pdftex}{color}
%\fi

%\usepackage{diagrams}
\usepackage{color}
\newcommand\coloremph[2][red]{\textcolor{#1}{\emph{#2}}}
\newcommand\norm[1]{\left\lVert #1 \right\rVert}
\newcommand\greenemph[2][green]{\textcolor{#1}{\emph{#2}}}
\newcommand{\EMP}[1]{\emph{\textcolor{red}{#1}}}

%\usepackage[all]{xy}
\usepackage{url}
\usepackage{verbatim}
\usepackage{latexsym}
\usepackage{amssymb,amstext,amsmath,amsthm}
\usepackage{epsf}
\usepackage{epsfig}
% \usepackage{isolatin1}
\usepackage{a4wide}
\usepackage{verbatim}
\usepackage{proof}
\usepackage{latexsym}
%\usepackage{mytheorems}
\newtheorem{theorem}{Theorem}[section]
\newtheorem{corollary}{Corollary}[section]
\newtheorem{lemma}{Lemma}[section]
\newtheorem{proposition}{Proposition}[section]


\usepackage{float}
\floatstyle{boxed}
\restylefloat{figure}


%%%%%%%%%
\def\oge{\leavevmode\raise
.3ex\hbox{$\scriptscriptstyle\langle\!\langle\,$}}
\def\feg{\leavevmode\raise
.3ex\hbox{$\scriptscriptstyle\,\rangle\!\rangle$}}

%%%%%%%%%


\newcommand\myfrac[2]{
 \begin{array}{c}
 #1 \\
 \hline \hline 
 #2
\end{array}}


\newcommand*{\Scale}[2][4]{\scalebox{#1}{$#2$}}%
\newcommand*{\Resize}[2]{\resizebox{#1}{!}{$#2$}}


\newcommand{\Pos}{\square}



\newcommand{\UU}{\mathsf{U}}
\newcommand{\II}{\mathbf{I}}

\newcommand{\Elem}{\mathsf{Elem}}


\newcommand{\id}{\mathsf{id}}
\newcommand{\pp}{\mathsf{p}}
\newcommand{\qq}{\mathsf{q}}

\newcommand{\mkbox}[1]{\ensuremath{#1}}
\newcommand{\ev}[1]{{\langle #1 \rangle}}
\newcommand{\Val}[1]{\ev{#1}}
\newcommand{\REIFY}{\mathsf{reify}}
\newcommand{\NF}{\mathsf{nf}}

\newcommand{\CC}{{\mathcal C}}
%\newcommand{\subst}{{\sf subst}}
\newcommand{\res}{{\sf res}}
\newcommand{\Int}{{\bf I}}
\newcommand{\PAIR}[1]{\langle #1\rangle}

\newcommand{\Sph}{{\sf S}^1}
\newcommand{\PROP}{{\sf prop}}
\newcommand{\SET}{{\sf set}}
\newcommand{\pair}[1]{{\langle #1 \rangle}}
\newcommand{\Prod}[2]{\displaystyle\prod _{#1}~#2}
\newcommand{\Sum}[2]{\displaystyle\sum _{#1}~#2}
\newcommand{\gothic}{\mathfrak}
\newcommand{\omicron}{*}
\newcommand{\gP}{{\gothic p}}
%\newcommand{\lift}[1]{\widetilde{#1}}
\newcommand{\lift}[1]{\overline{#1}}
\newcommand{\Prime}[1]{{#1}'}
\newcommand{\gM}{{\gothic M}}
\newcommand{\gN}{{\gothic N}}
\newcommand{\rats}{\mathbb{Q}}
\newcommand{\ints}{\mathbb{Z}}

% \usepackage{stmaryrd}
\newcommand{\abs}[2]{\lambda #1 . #2}            % abstraction of #1 in #2

%\newtheorem{proposition}[theorem]{Proposition}

%\documentstyle{article}
\setlength{\oddsidemargin}{0in} % so, left margin is 1in
\setlength{\textwidth}{6.27in} % so, right margin is 1in
\setlength{\topmargin}{0in} % so, top margin is 1in
\setlength{\headheight}{0in}
\setlength{\headsep}{0in}
\setlength{\textheight}{9.19in} % so, foot margin is 1.5in
\setlength{\footskip}{.8in}

% Definition of \placetitle
% Want to do an alternative which takes arguments
% for the names, authors etc.

\newcommand{\lbr}{\lbrack\!\lbrack}
\newcommand{\rbr}{\rbrack\!\rbrack}
\newcommand{\sem}[2] {\lbr #1 \rbr _{#2}}  % interpretation of the terms


%\newcommand{\APP}{\mathsf{app}}

\newcommand{\Type}{\mathsf{Type}}
\newcommand{\comp}{\mathsf{comp}}
\newcommand{\FILL}{\mathsf{fill}}
\newcommand{\horn}{\mathsf{horn}}

\newcommand{\Fib}{\mathsf{Fib}}
\newcommand{\isContr}{\mathsf{isContr}}
\newcommand{\Equiv}{\mathsf{Equiv}}
\newcommand{\isEquiv}{\mathsf{isEquiv}}
\newcommand{\isEquivId}{\mathsf{isEquivId}}

\newcommand{\model}[1]{Psh(#1)}

\newcommand{\Path}{\mathsf{Path}}
\newcommand{\Ext}{\mathsf{Ext}}
\newcommand{\ext}{\mathsf{ext}}
\newcommand{\wid}{\mathsf{wid}}
\newcommand{\inc}{\mathsf{inc}}
\newcommand{\Comp}{\mathsf{Comp}}
\newcommand{\hcomp}{\mathsf{hcomp}}
\newcommand{\coe}{\mathsf{coe}}
% Marc's macros
\newcommand{\op}[1]{#1^\mathit{op}}

\begin{document}

\title{Implemntation of dependent type theory}

\author{}
\date{27/09/2023}
\maketitle

%\rightfooter{}

\section*{Introduction}

We try to describe what should be the values for an implementation of the poset model
and then what should be the main algorithms.

We write $\varphi,~\psi,\dots$ for the cofibrations, and we have a type of cofibrations $\Phi$.
We also have an interval type $\II$.

We have two levels of closures $(\lambda_{x:A}t)\rho$ where $t$ is a term and $\rho$
an environment, and $(\lambda_zv)\alpha$ where $v$ is a value and $\alpha$ an interval
substitution of the form $z_1=e_1,\dots,z_n=e_n$.


\medskip

Ideally, I would like an implementation with head linear reduction and with the new
representation of proposition truncation, as the data type generated by constructors
$\inc~u$ and $\ext~u~[\psi\mapsto v]$ (which is equal to $v$ on $\psi$).



\section*{Values}

$$
v~::=~k^v[\psi\mapsto v]~|~(\lambda_{x}t)\rho~|~(\lambda_z~v)\alpha~|~v,v~|~V~|~cv~|~hcv$$
$$
V~::=~\Pi~v~v~|~\Sigma~v~v~|~U~|~\Path~v~v~v~|~\Ext~v~[\psi\mapsto (v,w,w')]
$$
$$
cv~::=~\coe~r_0~r_1~v~|~
       \coe~r_0~r_1~(\lambda_z\Pi~v~v)\alpha~v_0~|~\coe~r_0~r_1~(\lambda_z\Sigma~v~v)\alpha~v_0
$$
$$
       hcv~::=~\hcomp~r_0~r_1~v~[\psi\mapsto v]~|~
       \hcomp~r_0~r_1~(\Pi~v~v)~[\psi\mapsto v]~v_0~|~\hcomp~r_0~r_1~(\Sigma~v~v)~[\psi\mapsto v]~v_0
$$

$$
k~::=~x~|~k~v~|~k~r~|~\coe~r_0~r_1~(\lambda_zk)\alpha~v_0~|~\hcomp~r_0~r_1~k~[\psi\mapsto v]~v_0~
     |~k.1~|~k.2
$$

$$
\rho~::=~()~|~D\rho~|~\rho,x=v~~~~~~~~~\alpha~::=~()~|~\alpha,z=r
$$

 Here $r$ is an interval (lattice) expression.

 We can choose to have interval expression as values.

 We define substitution on values; the main clauses are for $\rho\alpha$
$$
 ()\alpha  = ()~~~(\rho,x=v)\alpha = (\rho\alpha, x = v\alpha)~~~(D\rho)\alpha = D(\rho\alpha)~~~~
$$
and for $\beta\alpha$
$$
()\alpha = \alpha~~~~~~(\beta,x=e)\alpha = (\beta\alpha, x = e\alpha)
$$

$\alpha$ represents a map between two stages. The substitution $()$ corresponds to going from a
stage $X$ to a stage $X,z_1,\psi,z_2,\dots$ obtained by adding more interval variables and constraints.
A substitution $(z=r)$ should correspond to a substitution $X\rightarrow X,z,\psi$ so that
$\psi(z=r)$ is true at stage $X$.

\section*{Main functions}

The suggestion is to follow the algorithms for the cartesian version, but to use connections
for the fact that singleton are contractible. Actually, the contractibility of singleton is
also expressed by the $\hcomp$ function.
(There are two notions of contractibility: the one expressing that any partial element can be
extended to a total element, and the one coming from type theory with a center of contraction.)

 The main functions seem to be eval and application on values.

\medskip

These two functions on values have as parameter the stage of evaluation $X$ (which is a
presentation of a distributive lattice). 

\medskip

For instance when we evaluate $[\psi\mapsto t]$ at stage $X$ and environment $\rho$
we should evaluate $t$ at stage $X,\psi$ and environment $\rho$.

%In the case of evaluation of $x(\rho,x=u)$ at stage $X$ we may have to ``simplify'' $u$.

The stage can only be modified by adding fresh interval variables or adding new constraints.

\medskip

I explain application on values $w~u$.

If $w$ is a closure $(\lambda_{x:A}t)\rho$ then we evaluate $t(\rho,x=u)$.

If $w$ is a closure $(\lambda_zv)\alpha$ then $u$ is an interval expression $r$ and we evaluate
$v(\alpha,z=r)$.

If $w$ is of the form $\coe~r_0~r_1~L$ then $L$ is a line value. If it is not in the form
$(\lambda_zV)\alpha$ where $V$ is $\Pi,\Sigma$ or $\Path$ then we generate a fresh interval variable
$z$ and evaluate $L~z$ at the stage $X,z$ getting a value $V$. We put then $L$ in the form
$(\lambda_zV)()$.

%We do something similar if $w$ is of the form $\hcomp~r_0~r_1~A~[\psi\mapsto v]$.


\section*{Some examples}

We define an auxiliary function $\comp$
$$
\comp~r_0~r_1~L~[\psi\mapsto u]~~u_0 =
 \hcomp~r_0~r_1~(L~r_1)~[\psi\mapsto (\lambda_z\coe~z~r_1~L~(u~z))()]~(\coe~r_0~r_1~L~u_0)
$$

 This function takes an extra argument $X$ which is the stage at which we do the computation
 in order to be able to generate the fresh variable $z$.

Here are some clauses for $\coe$

\medskip

\begin{tabular}{lll}
$
  \coe~r_0~r_1~(\lambda_z\Pi~A~B)\alpha~u_0~a_1$ & = & \\
  $\coe~r_0~r_1~(\lambda_{z'}B~(\alpha,z=z')(\coe~r_1~z'~(\lambda_zA)\alpha~a_1))()~
  (u_0~(\coe~r_1~r_0~(\lambda_zA)\alpha~a_1))$ & & \\
  & & \\
  $(\coe~r_0~r_1~(\lambda_z\Sigma~A~B)\alpha~u_0).1$ &  = & \\
  $\coe~r_0~r_1~(\lambda_zA)\alpha~u_0.1$ & & \\
  & & \\
  $(\coe~r_0~r_1~(\lambda_z\Sigma~A~B)\alpha~u_0).2$ & = & \\
  $\coe~r_0~r_1~(\lambda_{z'}B~(\alpha,z=z')(\coe~r_0~z'~(\lambda_zA)\alpha~u_0.1))()~u_0.2$ & & \\
    & & \\
  $\coe~r_0~r_1~(\lambda_z\Path~A~a~b)\alpha~u_0~r$ & = & \\
  $\comp~r_0~r_1~(\lambda_zA)\alpha~[r=0\mapsto (\lambda_za)\alpha,r=1\mapsto (\lambda_zb)\alpha]~(u_0~r)$ & &
\end{tabular}

\medskip

The definition of $\hcomp~U$ will use the $\Ext$ (Glue) constructor and the most complex functions ar
$\coe$ and $\hcomp$ for $\Ext$ types.

\medskip

Here are some clauses for $\hcomp$.

\medskip

\begin{tabular}{lll}
$\hcomp~r_0~r_1~(\Pi~A~B)~[\psi\mapsto u]~u_0~a$ & = & \\
  $\hcomp~r_0~r_1~(B~a)~[\psi\mapsto (\lambda_z~(u~z~a))()](u_0~a)$ & &\\
  & & \\
  $(\hcomp~r_0~r_1~(\Sigma~A~B)~[\psi\mapsto u]~u_0).1$ &  = & \\
  $\hcomp~r_0~r_1~A~[\psi\mapsto u.1]~u_0.1$ & & \\
    & & \\
  $(\hcomp~r_0~r_1~(\Sigma~A~B)~[\psi\mapsto u]~u_0).2$ &  = & \\
  $\comp~r_0~r_1~(\lambda_zB~(\hcomp~r_0~z~[\psi\mapsto u.1]~u_0.1))()~
  [\psi\mapsto u.2]~u_0.2$ & & \\
    & & \\
  $\hcomp~r_0~r_1~(\Path~A~a~b)~[\psi\mapsto u]~u_0~r$ & = & \\
  $\hcomp~r_0~r_1~A~[\psi\mapsto (\lambda_zu~z~r)(),
                    r=0\mapsto (\lambda_za)(),r=1\mapsto (\lambda_zb)()]~(u_0~r)$ & &
\end{tabular}

\medskip


Here to simplify, we have written $u.1$ for $(\lambda_z(u~z).1)()$ and
$u.2$ for $(\lambda_z(u~z).2)()$.

\section*{Some combinators}


In the implementation of cubicaltt, it was found convenient to introduce some combinators
that are obtained as evaluation of terms. For instance we have

$$\id = (\lambda_{A}\lambda_{x}x)()$$

so that $\id~A$ is the identity function for $A$. We can also define

$$\Fib = (\lambda_{A}\lambda_{B}\lambda_{f}\lambda_{a}\Sigma_{b:B}\Path~A~(f~b)~a)()$$

so that $\Fib~A~B~w~u$ is the fiber of $w$ at $u$, and let $D$ be the definition

$$D = [\isContr: U\rightarrow U = \lambda_{A}\Sigma_{a}\Pi_{x}\Path~A~a~x]$$

so that $\isContr D~A$ is the value for the fact that $A$ is contractible.

\medskip

We also have

$$
\isEquiv = (\lambda_{A:U}\lambda_{B:U}\lambda_{f:B\rightarrow A}\Pi_{a:A}
\isContr(\Sigma_{b:B}\Path~A~(f~b)~a))D$$

so that $\isEquiv~A~B~w$ expresses that $w$ is an equivalence.
         
We shall also need a combinator expressing that the identity is an equivalence.
This is the proof that singleton are contractible, which is simple if we have connections.

$$
\isEquivId =
(\lambda_{A:U}\lambda_{a:A}((a,z.a),\lambda_{v:\Sigma_{x:A}\Path~A~a~x}(v.2,z'.v.2(z\wedge z'))))()
$$


If $c$ a value of type $\isContr~A$, so that $c$ is convertible to a pair $a_0,p$, with
$p~a$ being a path in $\Path~A~a_0~a$,
we can use this to define a function that extends a partial element of $A$ to a total element

$$
\wid~A~(a_0,p)~[\psi\mapsto u] = \hcomp~0~1~A~[\psi\mapsto p~u]~a_0
$$


\section*{Glue/Ext type}

A canonical type at stage $X$ can be of the form $E = \Ext~A~[\varphi\mapsto (B,w,w')]$
where $\varphi\neq 1$ at stage $X$ and we have $A = B$ on $\varphi$ and $w:B\rightarrow A$
and $w'$ a proof that $w$ is an equivalence. The elements of this type are pairs $(a,b)$ where
$a$ is in $A$ and $b$ in $B$ and $w~b= a$ on $\varphi$.

We have the function $\ext~w$ of type $E\rightarrow A$ which is defined by $\ext~w~(a,b) = a$
in such a way that $\ext~w:E\rightarrow A$ extends the given partial function $w:B\rightarrow A$.

\subsection*{Homogeneous composition}

$\hcomp~r_0~r_1~E~[\psi\mapsto u]~u_0$ is defined in the following way. First we can always write
$u = (a,b)$ and $u_0 = (a_0,b_0)$. The output should be $a_1,b_1$ where
$$
\tilde{b} = \lambda_z\hcomp~r_0~z~B~[\psi\mapsto b]~b_0~~~~~~~~b_1 = \tilde{b}~r_1
$$
and
$$
a_1 = \hcomp~r_0~r_1~A~[\psi\mapsto a,~\varphi\mapsto (\lambda_z w~(\tilde{b}~z))()]~a_0
$$
Note that we don't use $w'$ in this computation.

\subsection*{Coerce function for Glue type}

The last case is coerce for $E$ and $\hcomp$ for $U$.

The case $\coe~r_0~r_1~(\lambda_z E)()~(a_0,b_0)$ is the most complex one.

The value $w'$ should be a witness that $w$ is an equivalence. Using $w'(r_1)$ and the combinators
$\wid$ and $\Fib$, we can define a function $f_1$, defined for values at stage $X,\varphi$,
which takes as argument $a$ in $A(r_1)$ and $b$ with a path $\omega$ in $\Path~A(r_1)~(w(r_1)~b)~a$
only defined on $\psi\leqslant \varphi(r_1)$
and which produces as output a pair $\tilde{b},\tilde{\omega}$ on $\varphi(r_1)$
which extends $b,\omega$.

We start by computing $\delta = \forall_z\varphi$. Using $\coe$ for $A$ and $B$ we compute
$\tilde{a}$ line in $A$ which is $a_0$ at $r_0$ and $\tilde{b}$, defined on $\delta$, line in $B$
which is $b_0$ at $r_0$. Using then the type $\Path~A~(w~\tilde{b})~\tilde{a}$ we can compute
by $\coe$ an element in $\Path~A(r_1)~(w(r_1)~\tilde{b}(r_1))~\tilde{a}(r_1)$ on $\delta$.
Using the function $f_1$, we get an element $b_1$ in $B(r_1)$ and a path connecting $w(r_1)~b_1$ and
$\tilde{a}(r_1)$, furthermore such that $b_1 = b_0$ on $r_0=r_1$.
Using $\hcomp$ for $A(r_1)$ we then get an element $a_1$ in $A(r_1)$ which
extends $w(r_1)~b_1$ and is equal to $a_0$ on $r_0=r_1$. The pair $a_1,b_1$ is the value of
$\coe~r_0~r_1~(\lambda_z E)()~(a_0,b_0)$.

\subsection*{Homogeneous composition for universes}

The remaining case is $\hcomp~r_0~r_1~U~[\psi\mapsto A]~A_0$. For this we compute for $z$
a witness $w'(z)$ that $\coe~~z~r_0~A$ is an equivalence between $A(z)$ and $A(r_0)$.
For this, we define the line of types $L = \lambda_z\isEquiv~A(z)~A(r_0)~(\coe~z~r_0~A)$.
Note that $L(r_0)$ expresses that the identity function of $A(r_0)$ is an equivalence.
We have an element of $L(r_0)$ which is $\isEquivId~A(r_0)$. We define

$$
w'(z) = \coe~r_0~z~L~(\isEquivId~A(r_0))
$$

The result is then
$\Ext~A_0~[\psi\mapsto (A(r_1),\coe~r_0~r_1~A,w'(r_0)),r_0=r_1\mapsto (A_0,\id~{A_0},\isEquivId~A_0)]$

\subsection*{Univalence}

 Univalence can be formulated as the type $\Pi_{A:U}\isContr~(\Sigma_{X:U}\Equiv~X~A)$.

%% \begin{thebibliography}{10}
%% \bibliographystyle{myauthordate}

%% %% \bibitem[{CRS}]{CRS21}
%% %% Thierry Coquand, Fabian Ruch, and Christian Sattler.
%% %% \newblock Constructive sheaf models of type theory.
%% %% \newblock {\em Math. Struct. Comput. Sci.}, 31(9):979--1002, 2021.

%% \end{thebibliography}


\end{document}     
                                                                                  
 
